\section{研究经历}
\cvitem{在德国}{%
  \begin{description} \addtolength{\itemsep}{0.5\baselineskip}%
    \item[“通过FT-IR光谱显微成像技术检测人体尿液中的癌变细胞”]根据细胞的FT-IR光谱特征,用随机森林分类器预测未知的细胞类型。可用于癌细胞的机器诊断。
    \item[“为振动显微光谱图像分析选择最小冗余的波数”]对于高维数据,根据特征之间的冗余性,以非监督的方式作特征选择。~\cite{zhong:2014}
  \end{description}}
\cvitem{在上海}{%
\begin{description}%
  \item[“人体结肠FT-IR显微光谱图像的分割,注释和分类”]首先对图像的像素点进行层次聚类,得到聚类树。然后用一种新的Tree Assignment方法来计算树的最优分割方案。研究成果已发表~\cite{zhong:2013}。
\end{description}}
