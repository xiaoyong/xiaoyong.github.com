\section{研究经历}
\cvitem{在德国}{%
  \begin{description} \addtolength{\itemsep}{0.5\baselineskip}%
    \item[“通过FT-IR光谱显微成像技术检测人体尿液中的癌变细胞”]首先对同一个样本的H\&E染色图和FT-IR光谱图进行配准(刚性加相似变换)。然后用图像分割方法(阈值法加分水岭算法)识别出细胞。接着请病理学家根据H\&E染色图注释细胞类型,产生训练集。最后用随机森林分类器根据红外光谱特征预测未知的细胞类型。课题的最终目标是病变细胞,特别是癌细胞的机器诊断。我的贡献:1)图像处理算法的研发和改进 2)写了一个整合所有功能的图形界面软件。
    \item[“为振动显微光谱图像分析选择最小冗余的波数”]对于高光谱图像这类高维数据,以无监督的方式进行特征选择。选择的标准是使得特征之间的相关性(用互信息衡量)最小。此方法基于著名的mRMR算法。mRMR适用于有监督学习,我把它修改成了无监督学习。在模拟数据和真实数据上的实验表明,在降低数据维度的同时,不会影响甚至可以提高后续分类的准确率。
  \end{description}}
\cvitem{在上海}{%
\begin{description}%
  \item[“人体结肠FT-IR显微光谱图像的分割,注释和分类”]首先对光谱图像的像素点进行层次聚类,得到聚类树。通过砍树分割图像,然后对分割出的图像区域进行人工注释。得到训练集后,可用分类器识别未知图像中的组织构成。同时,还用了一种新的Tree Assignment方法来计算树的最优分割方案。基于Tree Assignment,系统而定量地验证了不同聚类方法的效果。研究成果已发表~\cite{zhong:2013a}。
\end{description}}
