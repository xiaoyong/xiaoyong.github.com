\section{研究经历}
\cvitem{在德国}{%
  \begin{description} \addtolength{\itemsep}{0.5\baselineskip}%
    \item[“通过FT-IR光谱显微成像技术检测人体尿液中的癌变细胞”]首先对同一个样本的H\&E染色图和FT-IR光谱图进行自动匹配,然后请病理学家注释图像中的细胞,产生训练集。根据训练集,用随机森林分类器预测未知的细胞类型。可用于癌细胞的机器诊断。
    \item[“为振动显微光谱图像分析选择最小冗余的波数”]对于高维数据,根据特征之间的冗余性,以非监督的方式作特征选择。此方法被用于CARS光谱图像,一方面可以减少实验中需要测量的光谱数量,另一方面可以提高分类的准确率。
  \end{description}}
\cvitem{在上海}{%
\begin{description}%
  \item[“人体结肠FT-IR显微光谱图像的分割,注释和分类”]首先对图像的像素点进行层次聚类,得到树状图。通过砍树分割图像,对分割出的图像区域进行人工注释。得到训练集后,可用机器学习方法识别未知图像中的组织构成。同时,还用了一种新的Tree
    Assignment方法来计算树的最优分割方案。基于Tree
    Assignment,系统而定量地验证了不同聚类方法的效果。研究成果已发表~\cite{zhong:2013a}。
\end{description}}
