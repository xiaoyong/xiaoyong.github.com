% !Mode:: "TeX:UTF-8"
%% start of file `template-zh.tex'.
%% Copyright 2006-2013 Xavier Danaux (xdanaux@gmail.com).
%
% This work may be distributed and/or modified under the
% conditions of the LaTeX Project Public License version 1.3c,
% available at http://www.latex-project.org/lppl/.


\documentclass[11pt,a4paper,sans]{moderncv}   % possible options include font size ('10pt', '11pt' and '12pt'), paper size ('a4paper', 'letterpaper', 'a5paper', 'legalpaper', 'executivepaper' and 'landscape') and font family ('sans' and 'roman')

% moderncv 主题
\moderncvstyle{classic}                        % 选项参数是 ‘casual’, ‘classic’, ‘oldstyle’ 和 ’banking’
\moderncvcolor{red}                          % 选项参数是 ‘blue’ (默认)、‘orange’、‘green’、‘red’、‘purple’ 和 ‘grey’
%\nopagenumbers{}                             % 消除注释以取消自动页码生成功能

% 字符编码
%\usepackage[utf8]{inputenc}                   % 替换你正在使用的编码

%%%%%%%%%%%%%%%%%%% Begin Customization %%%%%%%%%%%%%%%%%%%%
\usepackage{setspace} % 调整行距
% 中文支持
\usepackage{fontspec,xltxtra,xunicode}
\usepackage[slantfont,boldfont]{xeCJK}
% 设置中文字体
% ==========================================================
\setCJKmainfont[BoldFont=Adobe Heiti Std,ItalicFont=Adobe Kaiti Std]{Adobe Song Std}
\setCJKsansfont{Adobe Heiti Std}
\setCJKmonofont{Adobe Fangsong Std}

\setCJKfamilyfont{zhsong}{Adobe Song Std}
\setCJKfamilyfont{zhhei}{Adobe Heiti Std}
\setCJKfamilyfont{zhfs}{Adobe Fangsong Std}
\setCJKfamilyfont{zhkai}{Adobe Kaiti Std}

\newcommand*{\songti}{\CJKfamily{zhsong}} % 宋体
\newcommand*{\heiti}{\CJKfamily{zhhei}} % 黑体
\newcommand*{\kaishu}{\CJKfamily{zhkai}} % 楷书
\newcommand*{\fangsong}{\CJKfamily{zhfs}} % 仿宋

% 让汉化更彻底
\renewcommand*{\namefont}{\Huge\sffamily\mdseries\upshape} % 姓名用黑体
% cventry改用中文的标点符号
\renewcommand*{\cventry}[7][.25em]{%
  \cvitem[#1]{#2}{%
    {\slshape#3}%
    \ifthenelse{\equal{#4}{}}{}{,{\bfseries#4}}%
    \ifthenelse{\equal{#5}{}}{}{,#5}%
    \ifthenelse{\equal{#6}{}}{}{,#6}%
    \strut%
    \ifx&#7&%
      \else{\newline{}\begin{minipage}[t]{\linewidth}\small#7\end{minipage}}\fi}}
\renewcommand*{\refname}{发表论文} % Publications -> 发表论文
%%%%%%%%%%%%%%%%%%%% End Customization %%%%%%%%%%%%%%%%%%%%%

% My own commands
\newcommand*{\eduseparator}{{\color{gray}\hrule}} % 教育背景分隔符
\newcommand*{\hrefcolor}[2]{\href{#1}{\color{blue}#2}} % 创建链接

% 调整页面出血
\usepackage[scale=0.8]{geometry}
\setlength{\hintscolumnwidth}{2.7cm}           % 如果你希望改变日期栏的宽度

% 个人信息
\name{钟巧勇}{}
%\title{简历题目 (可选项)}                     % 可选项、如不需要可删除本行
\address{上海市岳阳路320号}{200031}            % 可选项、如不需要可删除本行
\phone[mobile]{150~2132~9454}              % 可选项、如不需要可删除本行
%\phone[fixed]{021~5492~0235}               % 可选项、如不需要可删除本行
%\phone[fax]{+3~(456)~789~012}                 % 可选项、如不需要可删除本行
\email{solary.sh@gmail.com}                    % 可选项、如不需要可删除本行
\homepage{xiaoyong.org}                  % 可选项、如不需要可删除本行
\extrainfo{1988年1月26日出生于浙江金华}                 % 可选项、如不需要可删除本行
\photo[48pt][0.4pt]{xiaoyong.jpg}                  % ‘64pt’是图片必须压缩至的高度、‘0.4pt‘是图片边框的宽度 (如不需要可调节至0pt)、’picture‘ 是图片文件的名字;可选项、如不需要可删除本行
%\quote{引言(可选项)}                          % 可选项、如不需要可删除本行

% 显示索引号;仅用于在简历中使用了引言
\makeatletter
\renewcommand*{\bibliographyitemlabel}{\@biblabel{\arabic{enumiv}}}
\makeatother

% 分类索引
%\usepackage{multibib}
%\newcites{book,misc}{{Books},{Others}}
%----------------------------------------------------------------------------------
%            内容
%----------------------------------------------------------------------------------
\begin{document}
\maketitle

% 减少标题下面的空白
\vspace{-6ex}

% 1.5倍行距
\onehalfspacing

\section{教育背景}
\cventry{2009年9月 -- 现在}{硕博连读研究生}{中国科学院上海生命科学研究院计算生物学研究所}{上海}{}{计算生物学专业,生物学中的模式识别研究组,生物医学图像处理方向} % 第3到第6编码可留白
\eduseparator
\cventry{2012年12月 -- 2013年9月}{联合培养博士生}{波鸿鲁尔大学}{德国波鸿}{}{生物物理系生物信息学研究组}
\eduseparator
\cventry{2011年6月 -- 2011年7月}{访问学生}{波鸿鲁尔大学}{德国波鸿}{}{生物物理系生物信息学研究组}
\eduseparator
\cventry{2005年9月 -- 2009年6月}{理学学士}{南京大学生命科学学院}{南京}{}{生物技术专业,生理学方向}

\section{在校研究}
\cvitem{在德国}{%
  \begin{description} \addtolength{\itemsep}{0.5\baselineskip}%
    \item[“通过FT-IR光谱显微成像技术检测人体尿液中的癌变细胞”]首先对同一个样本的H\&E染色图和FT-IR光谱图进行配准(刚性加相似变换)。然后用图像分割方法(阈值法加分水岭算法)识别出细胞。接着请病理学家根据H\&E染色图注释细胞类型,产生训练集。最后用随机森林分类器根据红外光谱特征预测未知的细胞类型。课题的最终目标是病变细胞,特别是癌细胞的机器诊断。我的贡献:1)图像处理算法的研发和改进 2)写了一个整合所有功能的图形界面软件UroCell。
    \item[“为振动显微光谱图像分析选择最小冗余的波数”]对于高光谱图像这类高维数据,以无监督的方式进行特征选择。选择的标准是使得特征之间的相关性(用互信息衡量)最小。此方法基于著名的mRMR算法。mRMR适用于有监督学习,我把它修改成了无监督学习。在模拟数据和真实数据上的实验表明,在降低数据维度的同时,不会影响甚至可以提高后续分类的准确率。论文已被ICNC 2014会议接收~\cite{zhong:2014}。
  \end{description}}
\cvitem{在上海}{%
\begin{description}%
  \item[“人体结肠FT-IR显微光谱图像的分割,注释和分类”]首先对光谱图像的像素点进行层次聚类,得到聚类树。通过砍树分割图像,然后对分割出的图像区域进行人工注释。得到训练集后,可用分类器识别未知图像中的组织构成。同时,还用了一种新的Tree Assignment方法来计算树的最优分割方案。基于Tree Assignment,系统而定量地验证了不同聚类方法的效果。研究成果已发表~\cite{zhong:2013}。
\end{description}}


\section{编程项目}
\cvitem{竞赛}{%
\begin{itemize}%
  \item[$\dagger$]
    2012年“有道难题”网易手机软件创新大赛,\emph{Candy}队队员,作品“\hrefcolor{https://github.com/xiaoyong/yirisanxing}{一日三省}”(Android平台,Java开发),获东部赛区三等奖
  \item[$\dagger$] RubyVSPython Planet Conquer 2012 April Contest,用Ruby编程,获得冠军
%  \item[$\dagger$] Morgan Stanley Code Storm 2011模拟市场交易比赛,\emph{Blue Moon}队队员,用Python编程,涉及算法实现和网络编程,最后排名11/20(上海交大赛区)
\end{itemize}}
%\cvitem{开源软件项目}{%
%\begin{description}%
%  \item[\hrefcolor{http://voodoo.xiaoyong.org/}{Voodoo}:
%    PICB文件搜索网站]采用Sinatra Web框架搭建,搜索后端基于mlocate程序。
%  \item 查看更多:\hrefcolor{https://github.com/xiaoyong}{xiaoyong@GitHub}
%\end{description}}

\section{获奖情况}
\cvitem{2013年}{中科院上海生命科学研究院三好学生}
\cvitem{2007年}{南京大学优秀学生}
\cvitem{2006年}{南京大学人民奖学金二等奖}

%\newpage
\section{语言技能}
\cvitem{英语}{\textbf{大学英语六级}水平,熟悉并适应英语工作环境}


\section{专业技能}
\subsection{学术}
\cvitem{}{研究方向是生物医学图像处理,掌握数字图像处理、模式识别、机器学习、统计学等方面的知识和技术。熟悉数据结构、算法设计与分析。}
\subsection{编程}
\cvdoubleitem{Matlab}{精通}{C/C++}{熟悉}
\cvdoubleitem{Python}{精通}{Shell (Bash)}{熟悉}
\cvitem{网页开发}{熟悉前端开发(HTML,CSS,JavaScript和jQuery等);了解基于Ruby的Web框架(Ruby on Rails和Sinatra),以及MySQL服务器的配置和使用。}
\subsection{计算机技能}
\cvitem{}{通过\textbf{江苏省高等学校计算机二级},熟悉Windows、Mac OS X和Linux操作系统,能熟练使用微软Office和{\LaTeX}排版软件。}


\section{兴趣爱好}
\renewcommand{\listitemsymbol}{-~}            % change the symbol for lists
\cvlistdoubleitem{篮球}{电影}
\cvlistdoubleitem{钓鱼}{阅读}

%\section{其他 1}
%\cvlistitem{项目 1}
%\cvlistitem{项目 2}
%\cvlistitem{项目 3}

%\renewcommand{\listitemsymbol}{-}             % 改变列表符号

%\section{其他 2}
%\cvlistdoubleitem{项目 1}{项目 4}
%\cvlistdoubleitem{项目 2}{项目 5\cite{book1}}
%\cvlistdoubleitem{项目 3}{} 
% 来自BibTeX文件但不使用multibib包的出版物
%\renewcommand*{\bibliographyitemlabel}{\@biblabel{\arabic{enumiv}}}% BibTeX的数字标签
\nocite{*}
\bibliographystyle{abbrv}
\bibliography{publications}                    % 'publications' 是BibTeX文件的文件名

% 来自BibTeX文件并使用multibib包的出版物
%\section{出版物}
%\nocitebook{book1,book2}
%\bibliographystylebook{plain}
%\bibliographybook{publications}               % 'publications' 是BibTeX文件的文件名
%\nocitemisc{misc1,misc2,misc3}
%\bibliographystylemisc{plain}
%\bibliographymisc{publications}               % 'publications' 是BibTeX文件的文件名

\clearpage
\end{document}


%% 文件结尾 `template-zh.tex'.
