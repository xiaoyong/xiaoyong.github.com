%% start of file `template-zh.tex'.
%% Copyright 2006-2013 Xavier Danaux (xdanaux@gmail.com).
%
% This work may be distributed and/or modified under the
% conditions of the LaTeX Project Public License version 1.3c,
% available at http://www.latex-project.org/lppl/.


\documentclass[11pt,a4paper,sans]{moderncv}   % possible options include font size ('10pt', '11pt' and '12pt'), paper size ('a4paper', 'letterpaper', 'a5paper', 'legalpaper', 'executivepaper' and 'landscape') and font family ('sans' and 'roman')

% moderncv 主题
\moderncvstyle{classic}                        % 选项参数是 ‘casual’, ‘classic’, ‘oldstyle’ 和 ’banking’
\moderncvcolor{blue}                          % 选项参数是 ‘blue’ (默认)、‘orange’、‘green’、‘red’、‘purple’ 和 ‘grey’
%\nopagenumbers{}                             % 消除注释以取消自动页码生成功能

% 字符编码
\usepackage[utf8]{inputenc}                   % 替换你正在使用的编码
\usepackage{CJKutf8}

% 调整页面出血
\usepackage[scale=0.85]{geometry}
%\setlength{\hintscolumnwidth}{3cm}           % 如果你希望改变日期栏的宽度

% 个人信息
\name{\huge钟巧勇}{}
%\title{简历题目 (可选项)}                     % 可选项、如不需要可删除本行
\address{上海市岳阳路320号}{200031}            % 可选项、如不需要可删除本行
\phone[mobile]{150~2132~9454}              % 可选项、如不需要可删除本行
\phone[fixed]{021~5492~0235}               % 可选项、如不需要可删除本行
%\phone[fax]{+3~(456)~789~012}                 % 可选项、如不需要可删除本行
\email{solary.sh@gmail.com}                    % 可选项、如不需要可删除本行
\homepage{xiaoyong.org}                  % 可选项、如不需要可删除本行
\extrainfo{1988年1月26日出生于浙江}                 % 可选项、如不需要可删除本行
\photo[56pt][0.4pt]{xiaoyong.jpg}                  % ‘64pt’是图片必须压缩至的高度、‘0.4pt‘是图片边框的宽度 (如不需要可调节至0pt)、’picture‘ 是图片文件的名字;可选项、如不需要可删除本行
%\quote{引言(可选项)}                          % 可选项、如不需要可删除本行

% 显示索引号;仅用于在简历中使用了引言
%\makeatletter
%\renewcommand*{\bibliographyitemlabel}{\@biblabel{\arabic{enumiv}}}
%\makeatother

% 分类索引
%\usepackage{multibib}
%\newcites{book,misc}{{Books},{Others}}
%----------------------------------------------------------------------------------
%            内容
%----------------------------------------------------------------------------------
\begin{document}
\begin{CJK}{UTF8}{gbsn}                       % 详情参阅CJK文件包
\maketitle

\section{教育背景}
\cventry{2009年9月 --
现在}{硕博连读研究生}{中国科学院上海生命科学研究院计算生物学研究所}{上海}{}{计算生物学专业,生物中的模式识别研究组,生物医学图像处理方向}
% 第3到第6编码可留白
\cventry{2005年9月 --
2009年6月}{理学学士}{南京大学生命科学学院}{南京}{}{生物技术专业,生理学方向}


\section{学习和工作经历}
\subsection{留学经历}
\cventry{2012年12月 -- 2013年9月}{联合培养博士生}{波鸿鲁尔大学}{德国波鸿}{}{生物物理系生物信息学研究组\newline{}%
  研究课题:
  \begin{description}%
    \item[“通过FT-IR光谱显微成像技术检测人体尿液中的癌变细胞”]首先对同一个样本的HE染色图和FT-IR光谱图进行自动匹配,然后请病理学家注释图像中的细胞,产生训练集。根据训练集,用随机森林分类器预测未知的细胞类型。可用于癌细胞的机器诊断。
    \item[“为振动显微光谱图像分析选择最小冗余的波数”]对于高维数据,根据特征之间的冗余性,以非监督的方式作特征选择。此方法被用于CARS光谱图像,一方面可以减少实验中需要测量的光谱数量,另一方面可以提高分类的准确率。
  \end{description}
  }
\cventry{2011年6月 -- 2011年7月}{访问学生}{波鸿鲁尔大学}{德国波鸿}{}{生物物理系生物信息学研究组}
\subsection{学术研究}
\cvitem{博士研究\\课题}{%
\begin{description}%
  \item[“人体结肠FT-IR显微光谱图像的分割,注释和分类”]首先对图像的像素点进行层次聚类,得到树状图。通过砍树分割图像,对分割出的图像区域进行人工注释。得到训练集后,可用机器学习方法识别未知图像中的组织构成。同时,我们还用了一种新的Tree
    Assignment方法来计算树的最优分割方案。基于Tree
    Assignment,我们系统而定量地验证了不同聚类方法的效果。
\end{description}}
\cvitem{学士论文}{%
\begin{description}%
  \item[“人类转录因子结合位点的预测”]首先通过WordSpy算法找到启动子序列中可能的转录因子结合位点,然后与TRANSFAC数据库中的模体作序列匹配,以此找到可靠的转录因子结合位点。
\end{description}}
\subsection{编程}
\cvitem{竞赛}{%
\begin{itemize}%
  \item[$\dagger$] 2012年“有道难题”网易手机软件创新大赛,\emph{Candy}队队员,作品“{\href{https://github.com/xiaoyong/yirisanxing}{\color{cyan}一日三省}}”,东部赛区三等奖
  \item[$\dagger$] RubyVSPython Planet Conquer 2012 April Contest,用Ruby获得冠军
  \item[$\dagger$] Morgan Stanley Code Storm 2011,\emph{Blue Moon}队队员,排名11/20(上海交大赛区)
\end{itemize}}
\cvitem{开源软件\\项目}{%
\begin{description}%
  \item[\href{http://voodoo.xiaoyong.org/}{\color{cyan}Voodoo}:
    PICB文件搜索网站]:网站用Sinatra Web框架搭建,搜索后端基于mlocate程序。
  \item 查看更多:\url{https://github.com/xiaoyong}
\end{description}}


%\newpage
\section{语言技能}
\cvitemwithcomment{英语}{大学英语六级水平}{熟悉并适应英语工作环境}


\section{职业技能}
\subsection{学术}
\cvitem{简介}{研究方向是生物医学图像处理,熟悉数字图象处理、模式识别、机器学习、统计学等方面的知识和技术。}
\subsection{编程}
\cvdoubleitem{Matlab}{精通}{C/C++}{熟练}
\cvdoubleitem{Ruby}{精通}{Shell (Bash)}{熟练}
\cvitem{网页开发}{熟悉HTML,CSS和JavaScript;了解基于Ruby的Web框架(Ruby on Rails和Sinatra)}
\subsection{计算机}
\cvdoubleitem{操作系统}{精通Windows, Mac OS X和Linux的使用}{办公和排版}{熟练使用微软Office和{\LaTeX}}
%\cvdoubleitem{Text Editing}{Expert on Vim}{}{}


\section{兴趣爱好}
\renewcommand{\listitemsymbol}{-~}            % change the symbol for lists
\cvlistdoubleitem{篮球}{电影}

%\section{其他 1}
%\cvlistitem{项目 1}
%\cvlistitem{项目 2}
%\cvlistitem{项目 3}

%\renewcommand{\listitemsymbol}{-}             % 改变列表符号

%\section{其他 2}
%\cvlistdoubleitem{项目 1}{项目 4}
%\cvlistdoubleitem{项目 2}{项目 5\cite{book1}}
%\cvlistdoubleitem{项目 3}{} 
% 来自BibTeX文件但不使用multibib包的出版物
%\renewcommand*{\bibliographyitemlabel}{\@biblabel{\arabic{enumiv}}}% BibTeX的数字标签
\nocite{*}
\bibliographystyle{plain}
\bibliography{publications}                    % 'publications' 是BibTeX文件的文件名

% 来自BibTeX文件并使用multibib包的出版物
%\section{出版物}
%\nocitebook{book1,book2}
%\bibliographystylebook{plain}
%\bibliographybook{publications}               % 'publications' 是BibTeX文件的文件名
%\nocitemisc{misc1,misc2,misc3}
%\bibliographystylemisc{plain}
%\bibliographymisc{publications}               % 'publications' 是BibTeX文件的文件名

\clearpage\end{CJK}
\end{document}


%% 文件结尾 `template-zh.tex'.
